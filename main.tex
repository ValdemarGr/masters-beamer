\documentclass{beamer}

\usefonttheme[onlymath]{serif}
\usepackage[utf8]{inputenc}
\usepackage{amsfonts}
\usepackage{bussproofs}
\usepackage{amsmath}
\usepackage{caption}
\usepackage{subcaption}
\usepackage[style=alphabetic]{biblatex}

\addbibresource{bib.bib}
%Information to be included in the title page:
\title{Aspects of efficiency in functional programming languages}
\author{Samuel Valdemar Grange}
\institute{Department of Mathematics and Computer Science (IMADA)}
\date{}

\begin{document}

\frame{\titlepage}

\begin{frame}
\frametitle{Programming languages}
\begin{itemize}
    \item The basis of functional programming languages.
    \begin{itemize}
        \item High-level programming language.
    \end{itemize}
    \item Recursion and lambda lifting.
    \item Runtime systems.\\
    \texttt{fun f x = x + (f (x - 1)); fun main = f 5;}\\
    $\Rightarrow$ $\texttt{let } f = \lambda x.x + (\textit{f } (x - 1)) \texttt{ in } \textit{f } 5$
    \\or\\
    $\Rightarrow$ $\texttt{let } f' = (\lambda f''. \texttt{let } f = f'' f'' \texttt{ in }$\\
    \hspace*{1cm}$\texttt{let } f = \lambda x.x + (\textit{f } (x - 1))$\\
    $) \texttt{ in } \texttt{let } f = f' f' \texttt{ in } \textit{f } 5$
\end{itemize}
\end{frame}

\begin{frame}
\frametitle{Types}
\begin{columns}
    \begin{column}{0.6\textwidth}
        \begin{itemize}
            \item Mono, poly and environment.
            \item Generalization and instantitation.
            \item Let polymorphism and type hierachy ($\sqsubseteq$).
        \end{itemize}
    \end{column}
    
    \begin{column}{0.4\textwidth}
        \begin{align}
        \tau &::= a \,|\, \tau \rightarrow \tau \,|\, C\tau_1 \dots \tau_n \tag*{}\\
        \sigma &::= \tau \,|\, \forall a.\sigma \tag*{}\\
        \Gamma &::= \epsilon \,|\, \Gamma, x: \sigma \tag*{}
        \end{align}
    \end{column}
\end{columns}
\end{frame}

\begin{frame}
\frametitle{Reconstruction}
    \begin{itemize}
        \item Checking is top-down, inference is bottom-up.
        \item Begin at leaf; variable (introduced by Abs) or number, then go back up while gathering constraints.
        \item Constraints generated from \textit{most general unifier} in the form of substitutions.
        \item Constraints are only imposed on non-bound type variables.
        \item Generalization over (locally) free type variables and instantitation over quantified type variables.
    \end{itemize}
\end{frame}

\begin{frame}
\frametitle{Polymorphism}
\begin{itemize}
    \item Let vs Abs + App; $\texttt{let } x = e \texttt{ in } y \Leftrightarrow (\lambda x. y) e$?
    \item Abs (potentially) introduces polymorphic types and App must accept polymorphic parameters (even themselves $\Leftrightarrow$ polymorphic recursion).
\end{itemize}
\begin{figure}
    \centering
    \begin{subfigure}[b]{0.4\textwidth}
    \centering
    \begin{prooftree}
        \AxiomC{$\Gamma, x: \tau_1 \vdash e: \tau_2$}
        \LeftLabel{Abs}
        \UnaryInfC{$\Gamma \vdash \lambda x.e : \tau_1 \rightarrow \tau_2$}
    \end{prooftree}
    \caption{Abs in Hindley-Milner}
    \end{subfigure}
    \begin{subfigure}[b]{0.4\textwidth}
    \centering
    \begin{prooftree}
        \AxiomC{$\Gamma, x: \sigma \vdash e: \tau$}
        \LeftLabel{Abs}
        \UnaryInfC{$\Gamma \vdash \lambda x.e : \sigma \rightarrow \tau$}
    \end{prooftree}
    \caption{Abs in System F~\cite{WELLS1999111}}
    \end{subfigure}

    \begin{subfigure}[b]{1\textwidth}
    \centering
    \begin{prooftree}
        \AxiomC{$\Gamma \vdash e_1: \tau_1 \rightarrow \tau_2$}
        \AxiomC{$\Gamma \vdash e_2: \tau_1$}
        \LeftLabel{App}
        \BinaryInfC{$\Gamma \vdash e_1 e_2 : \tau_2$}
    \end{prooftree}
    \caption{App in Hindley-Milner}
    \end{subfigure}
    \begin{subfigure}[b]{1\textwidth}
    \centering
    \begin{prooftree}
        \AxiomC{$\Gamma \vdash e_1: \sigma \rightarrow \tau$}
        \AxiomC{$\Gamma \vdash e_2: \sigma$}
        \LeftLabel{App}
        \BinaryInfC{$\Gamma \vdash e_1 e_2 : \tau$}
    \end{prooftree}
    \caption{App in System F~\cite{WELLS1999111}}
    \end{subfigure}
\end{figure}
\end{frame}

\begin{frame}
\frametitle{Polymorphism cont.}
    \begin{itemize}
        \item How does one unify to polymorphic types?.
        \item Boils down to a problem named semi-unification, which is undecidable~\cite{WELLS1999111,NAMEHERE}.
    \end{itemize}
\end{frame}

\begin{frame}
\frametitle{Errata}
    \begin{itemize}
        \item "Considerations of functional programming language implmentations."
        \item Multi let and mutual recursion (the reconstruction algorithm being trivial).
    \end{itemize}
    \begin{figure}
    \hspace*{-1.5cm}
    \begin{subfigure}[b]{1\textwidth}
    \begin{prooftree}
        \AxiomC{$\Gamma, x_1: \tau_1 \dots, x_n: \tau_n \vdash e_1: \sigma_1, \dots e_n: \sigma_n$}
        \AxiomC{$\Gamma, x_1: \sigma_1 \dots, x_n: \sigma_n  \vdash e: \tau$}
        \BinaryInfC{$\Gamma \vdash \texttt{let } \{x_1 = e_1, \dots x_n = e_n\} \texttt{ in } e: \tau$}
    \end{prooftree}
    \caption{Multi Let proof rule.}
    \end{subfigure}
    \end{figure}
\end{frame}

\begin{frame}
    \printbibliography
\end{frame}

\end{document}