\documentclass{beamer}

\usefonttheme[onlymath]{serif}
\usepackage[utf8]{inputenc}
\usepackage{amsfonts}
\usepackage{bussproofs}
\usepackage{amsmath}
\usepackage{caption}
\usepackage{subcaption}
\usepackage[style=alphabetic]{biblatex}

\addbibresource{bib.bib}
%Information to be included in the title page:
\title{Aspects of efficiency in functional programming languages}
\author{Samuel Valdemar Grange}
\institute{Department of Mathematics and Computer Science (IMADA)}
\date{}

\begin{document}

\frame{\titlepage}

\begin{frame}
\frametitle{Programming languages}
\begin{itemize}
    \item The basis of functional programming languages.
    \begin{itemize}
        \item High-level programming language.
    \end{itemize}
    \item Recursion and lambda lifting.
    \item Runtime systems.\\
    \texttt{fun f x = f x;}\\
    $\Rightarrow$ $\texttt{let } f = \lambda x.f x \texttt{ in } \dots$
    \\or\\
    $\Rightarrow$ $\texttt{let } f' = (\lambda f''. \texttt{let } f = f'' f'' \texttt{ in }$\\
    \hspace*{1cm}$\texttt{let } f = \lambda x.f x$\\
    $) \texttt{ in } \texttt{let } f = f' f' \texttt{ in } \dots$
\end{itemize}
\end{frame}

\begin{frame}
\frametitle{Types}
\begin{columns}
    \begin{column}{0.6\textwidth}
        \begin{itemize}
            \item Minimal type system.\\
            $\texttt{let } id = \lambda x.x \texttt{ in } \dots$\\
            where $id: a \rightarrow a$ then $(id \,\, id) 5$?
            \item Mono, poly and environment.
            \item Generalization and instantitation.
            \item Let polymorphism, recursion and type hierachy ($\sqsubseteq$).
        \end{itemize}
    \end{column}
    
    \begin{column}{0.4\textwidth}
        \begin{align}
        \tau &::= a \,|\, \tau \rightarrow \tau \,|\, C\tau_1 \dots \tau_n \tag*{}\\
        \sigma &::= \tau \,|\, \forall a.\sigma \tag*{}\\
        \Gamma &::= \epsilon \,|\, \Gamma, x: \sigma \tag*{}
        \end{align}
    \end{column}
\end{columns}
\end{frame}

\begin{frame}
\frametitle{Hindley-Milner}
\begin{figure}
            \begin{prooftree}
                \AxiomC{$x: \sigma \in \Gamma$}
                \LeftLabel{Var}
                \UnaryInfC{$\Gamma\vdash x:\sigma$}
            \end{prooftree}
            \begin{prooftree}
                \AxiomC{$\Gamma \vdash e_1 : \tau_1 \rightarrow \tau_2$}
                \LeftLabel{App}
                \AxiomC{$\Gamma \vdash e_2 : \tau_1$}
                \BinaryInfC{$\Gamma \vdash e_1 e_2 : \tau_2$}
            \end{prooftree}
    
            \begin{prooftree}
                \AxiomC{$\Gamma, x: \tau_1 \vdash e : \tau_2$}
                \LeftLabel{Abs}
                \UnaryInfC{$\Gamma \vdash \lambda x . e : \tau_1 \rightarrow \tau_2$}
            \end{prooftree}
            \begin{prooftree}
                \AxiomC{$\Gamma \vdash e_1 : \sigma$}
                \LeftLabel{Let}
                \AxiomC{$\Gamma ,x : \sigma \vdash e_2 : \tau$}
                \BinaryInfC{$\Gamma \vdash \texttt{let } x = e_1 \texttt{ in } e_2 : \tau$}
            \end{prooftree}
    
            \begin{prooftree}
                \AxiomC{$\Gamma \vdash e : \sigma_1$}
                \AxiomC{$\sigma_1 \sqsubseteq \sigma_2$}
                \LeftLabel{Inst}
                \BinaryInfC{$\Gamma \vdash e : \sigma_2$}
            \end{prooftree}
            \begin{prooftree}
                \AxiomC{$\Gamma \vdash e : \sigma$}
                \AxiomC{$a \notin \textit{free}(\Gamma)$}
                \LeftLabel{Gen}
                \BinaryInfC{$\Gamma \vdash e : \forall a . \sigma$}
            \end{prooftree}
        \caption{Hindley-Milner type rules}
        \label{fig:hmrules}
    \end{figure}
\end{frame}

\begin{frame}
\frametitle{Reconstruction}
    \begin{itemize}
        \item Checking is top-down, inference is bottom-up.
        \item Begin at leaf; variable (introduced by Abs) or number, then go back up while gathering constraints.
        \item Constraints generated from \textit{most general unifier} in the form of substitutions.
        \item Constraints are only imposed on non-bound type variables.
        \item Generalization over (locally) free type variables and instantitation over quantified type variables.
    \end{itemize}
\end{frame}

\begin{frame}
\frametitle{Damas-Milner}
\begin{figure}
    
            \begin{prooftree}
                \AxiomC{$x: \sigma \in \Gamma$}
                \AxiomC{$\tau = \textit{inst}(\sigma)$}
                \LeftLabel{Var}
                \BinaryInfC{$\Gamma \vdash x:\tau , \emptyset$}
            \end{prooftree}
            \begin{prooftree}
                \AxiomC{$\tau_1 = \textit{fresh}$}
                \AxiomC{$\Gamma, x: \tau_1 \vdash e: \tau_2, S$}
                \LeftLabel{Abs}
                \BinaryInfC{$\Gamma \vdash \lambda x . e : S\tau_1 \rightarrow \tau_2, S$}
            \end{prooftree}
    
            \begin{center}
                App
            \end{center}
            \vspace{-0.7cm}
            \begin{prooftree}
                \AxiomC{$\Gamma \vdash e_1 : \tau_1, S_1 \,\, \tau_3 = \textit{fresh}$}
                \AxiomC{$S_1 \Gamma \vdash e_2 : \tau_2, S_2 \,\, S_3 = \textit{unify}(S_2 \tau_1, \tau_2 \rightarrow \tau_3)$}
                \BinaryInfC{$\Gamma \vdash e_1 e_2 : S_3 \tau_3, S_3 \cdot S_2 \cdot S_1$}
            \end{prooftree}
    
            \begin{prooftree}
                \AxiomC{$\Gamma \vdash e_1 : \tau_1, S_1$}
                \AxiomC{$S_1 \Gamma , x : (S_1 \Gamma)(\tau_1) \vdash e_2 : \tau_2, S_2$}
                \LeftLabel{Let}
                \BinaryInfC{$\Gamma \vdash \texttt{let } x = e_1 \texttt{ in } e_2 : \tau_2, S_2 \cdot S_1$}
            \end{prooftree}
      \caption{Algorithm W. Note that $\Gamma(\tau)$ means the generalization of $\tau$ under $\Gamma$.}
        \label{fig:dmrules}
    \end{figure}
\end{frame}

\begin{frame}
\frametitle{Polymorphism}
\begin{itemize}
    \item Let vs Abs + App; $\texttt{let } x = e \texttt{ in } y \Leftrightarrow (\lambda x. y) e$?
    \item Abs (potentially) introduces polymorphic types and App must accept polymorphic parameters (even themselves $\Leftrightarrow$ polymorphic recursion).
\end{itemize}
\begin{figure}
    \centering
    \begin{subfigure}[b]{0.4\textwidth}
    \centering
    \begin{prooftree}
        \AxiomC{$\Gamma, x: \tau_1 \vdash e: \tau_2$}
        \LeftLabel{Abs}
        \UnaryInfC{$\Gamma \vdash \lambda x.e : \tau_1 \rightarrow \tau_2$}
    \end{prooftree}
    \caption{Abs in Hindley-Milner}
    \end{subfigure}
    \begin{subfigure}[b]{0.4\textwidth}
    \centering
    \begin{prooftree}
        \AxiomC{$\Gamma, x: \sigma \vdash e: \tau$}
        \LeftLabel{Abs}
        \UnaryInfC{$\Gamma \vdash \lambda x.e : \sigma \rightarrow \tau$}
    \end{prooftree}
    \caption{Abs in System F~\cite{WELLS1999111}}
    \end{subfigure}

    \begin{subfigure}[b]{1\textwidth}
    \centering
    \begin{prooftree}
        \AxiomC{$\Gamma \vdash e_1: \tau_1 \rightarrow \tau_2$}
        \AxiomC{$\Gamma \vdash e_2: \tau_1$}
        \LeftLabel{App}
        \BinaryInfC{$\Gamma \vdash e_1 e_2 : \tau_2$}
    \end{prooftree}
    \caption{App in Hindley-Milner}
    \end{subfigure}
    \begin{subfigure}[b]{1\textwidth}
    \centering
    \begin{prooftree}
        \AxiomC{$\Gamma \vdash e_1: \sigma \rightarrow \tau$}
        \AxiomC{$\Gamma \vdash e_2: \sigma$}
        \LeftLabel{App}
        \BinaryInfC{$\Gamma \vdash e_1 e_2 : \tau$}
    \end{prooftree}
    \caption{App in System F~\cite{WELLS1999111}}
    \end{subfigure}
\end{figure}
\end{frame}

\begin{frame}
\frametitle{Polymorphism cont.}
    \begin{itemize}
        \item How does one unify to polymorphic types?.
        \item Boils down to a problem named semi-unification, which is undecidable~\cite{WELLS1999111,NAMEHERE}.
    \end{itemize}
\end{frame}

\begin{frame}
\frametitle{Evaluation}
    \begin{itemize}
        \item Evaluation by substitution.
        \item Call by name and call by value.
        \item Storing labelled expressions.
        \item Freshness ($k \notin \textit{bound}(e)$ and clearly $k \notin \textit{free}(e)$).
    \end{itemize}
\end{frame}


\begin{frame}
\frametitle{Errata}
    \begin{itemize}
        \item "Considerations of functional programming language implmentations."
        \item Case instead of Scott encoding.
        \item Multi let for shared scopes (mutual recursion) (the reconstruction algorithm).
    \end{itemize}
    \begin{figure}
    \hspace*{-1.5cm}
    \begin{subfigure}[b]{1\textwidth}
    \begin{prooftree}
        \AxiomC{$\Gamma, x_1: \tau_1 \dots, x_n: \tau_n \vdash e_1: \sigma_1, \dots e_n: \sigma_n$}
        \AxiomC{$\Gamma, x_1: \sigma_1 \dots, x_n: \sigma_n  \vdash e: \tau$}
        \BinaryInfC{$\Gamma \vdash \texttt{let } \{x_1 = e_1, \dots x_n = e_n\} \texttt{ in } e: \tau$}
    \end{prooftree}
    \caption{Multi Let proof rule.}
    \end{subfigure}
    \end{figure}
\end{frame}

\begin{frame}
  \frametitle{Errata}
    \begin{itemize}
      \item Type constructors and sum-types missing (implemented in the fixes branch as described in the thesis).\\
        \vspace*{0.5cm}
        \texttt{type Stack a = | Nil}\\
        \texttt{-| Cons a (List a);}\\
        \texttt{+| Cons a (Stack a);}\\
        \vspace*{0.5cm}
        \texttt{-introduced Cons with type $\forall$c0.$\forall$d0.$\forall$e0.$\forall$g0.(c0 -> (d0 -> (e0 -> ((c0 -> (d0 -> g0)) -> g0))))}\\
        \texttt{+introduced Cons with type $\forall$adt\_f0.(adt\_f0 -> ((ADT(List) adt\_f0) -> (ADT(List) adt\_f0)))}\\
        \vspace*{0.5cm}
        \texttt{-applied ((Cons 1) ((Cons 2) ((Cons 3) Nil))) with type ((Integer ) -> (((Integer ) -> ((h1 -> (((Integer ) -> ((n1 -> (((Integer ) -> ((p1 -> (q1 -> p1)) -> o1)) -> o1)) -> i1)) -> i1)) -> (Integer ))) -> (Integer )))}
    \end{itemize}
\end{frame}

\begin{frame}
    \printbibliography
\end{frame}

\end{document}
